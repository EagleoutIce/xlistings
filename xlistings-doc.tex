\errorcontextlines=99999
\documentclass[10pt,DIV=12]{scrartcl}

\usepackage[T1]{fontenc}
\usepackage[utf8]{inputenc}

\usepackage{xcolor}
\usepackage{parskip}
\usepackage{enumitem}

\usepackage[sfdefault,scale=.9]{FiraSans}
% fitting math font
\usepackage{newtxsf}
\usepackage{inconsolata}
\usepackage{csquotes}
\usepackage{multicol}
\usepackage{pgffor}

\usepackage[style={plain number},extendednums]{xlistings}
\LoadLanguages{latex,java,bash,rubberduck,xml,json}

\usepackage{microtype}
\usepackage[colorlinks=true,breaklinks=true,allcolors=cyan!65!black]{hyperref}
\usepackage{cleveref}
\usepackage{footnotebackref}

\xlstlangoverride{latex}{morekeywords=[5]{\\xlstsetstyle,\\xlstlangoverride,\\LoadLanguages,\\xlstblacklistlinenumbers,\\lstcolorlet,\\xlstDefineStyles}}
% \xlstlangoverride*{latex}{morekeywords=[2]{xlistings}}

\let\T\texttt
\def\argref#1#2{\hyperref[arg:#1]{\T{#2}}}
\def\cmdref#1{\hyperref[arg:#1]{\T{\xlstGetStyle{command}\textbackslash #1}}}

\title{\T{xlistings} v0.1.0}
\subtitle{opinionated \T{listings} extensions}
\date{\today}
\author{Florian Sihler}

\begin{document}
    \maketitle

\section{Introduction}

This package extends on the \href{https://ctan.org/pkg/listings}{listings} package, providing an easier front-end to create code blocks of selected languages, support for number highlighting, segment highlighting, non-selectable line numbers,\footnote{If a number is truly non-selectable depends on the viewer used. To ensure that they can not be selected would require images, a process we currently do not support.} and more.
This package is not compatible with the \href{https://ctan.org/pkg/minted}{minted} package.
% , it provides a similar interface for code highlighting that can be used as a (currently partial) drop-in replacement using the  option.
In summary this package provides the following improvements over the \T{listings} package:

\begin{itemize}[nosep,itemsep=2pt,leftmargin=*]
    \item Highlighting of numbers in code blocks: \bjava{10_234 + x1 * 0x34 - x2}\; (\argref{hlnumbers}{hlnumbers} and \argref{extendednums}{extendednums} options)
    \item Support for the \blatex{\\begin\{minted\}\{<lang>\} :ldots: \\end\{minted\}} environment (\argref{fakeminted}{fakeminted} option)
    \item Wrapper macros like \blatex{\\bjava\{int i\}} and \clatex{\\cjava\{int i\}} and environments \blatex{\\begin\{plainjava\}}
    \item Language sensitive override: \blatex{\\xlstlangoverride\{latex\}\{morekeywords=[5]\{\\xlstsetstyle\}\}}
    \item Support for (\href{https://ctan.org/pkg/accsupp}{accsupp} based) non-selectable line numbers and characters:
\begin{minted}[xleftmargin=12pt]{java}
System.out.println("Hello, World!");
\end{minted}
    \item Support for blacklisting line numbers with \cmdref{xlstblacklistlinenumbers}
    \item Support for umlauts and UTF-8 encoding (with the \href{https://ctan.org/pkg/listingsutf8
    }{listingsutf8} package)
    \item Provides \texttt{autogobble} to remove leading spaces (with the \href{https://ctan.org/pkg/lstautogobble}{lstautogobble} package)
    \item Comfort key \blatex{add to literate} to add elements to the literate list
    \item \blatex{\\LoadLanguages\{<lang>\}} to load a language or multiple languages on demand
    \item Opinionated language overwrites (see the \T{langs/} folder)
    \item Opinionated default literates such as \T{:ldots:}~(\bjava{:ldots:}), \T{:lan:}~(\bjava{:lan:}), \T{:to:}~(\bjava{:to:}) and \T{:c:}~(\bjava{:c:}, \textit{breaks highlighting})
\end{itemize}

\subsection{Loading the Package}\label{sec:loading-pkg}

To load the package, simply use:

\xlstsetstyle{default}
\begin{minted}{latex}
\usepackage{xlistings}
\end{minted}

Throughout this document, we will use the \T{xlistings} package to highlight code.
If you are no fan of the \enquote{bordered} look of the code blocks, you can use \blatex{\\xlstsetstyle\{plain number\}} to switch the look and feel, or use the \argref{style}{style} option of the package:

\xlstsetstyle{plain number}
\begin{minted}{latex}
\usepackage[style={plain number}]{xlistings}
\end{minted}

Or, using \blatex{\\xlstsetstyle\{plain\}}:

\xlstsetstyle{plain}
\begin{minted}{latex}
\usepackage[style=plain]{xlistings}
\end{minted}
% TODO: provide a package option!

\subsection{Origin}

Originally, this package was part of \href{https://github.com/EagleoutIce/LILLY}{\MakeUppercase{lilly}}, which I again ported to the \href{https://github.com/EagleoutIce/sopra-collection}{sopra-collection} during my studies. This package is the standalone and improved version of these packages. The process is work in progress and I welcome any kind of feedback.

\subsection{Dependencies}

This package imports the following packages:\vspace*{-.5\baselineskip}
\begin{multicols}{3}
\begin{itemize}[nosep]
    \def\pkgparse#1:#2\@nil{%
        \T{#1}\ifx!#2!\else\textsuperscript{(\textit{#2})}\fi%
    }
    \foreach \pkg in {%
        kvoptions:,
        needspace:{\argref{guardspace}{guardspace}},%
        etoolbox:,xcolor:,pgfkeys:,lstautogobble:,listingsutf8:try,listings:,accsupp:try,
        tikz:{\argref{highlights}{highlights}}%
    } {
        \item \expandafter\pkgparse\pkg\@nil
    }
\end{itemize}
\footnotesize\textit{\enquote{try} notes that the package is only loaded if present and may rely on a fallback if not. For example, \T{accsupp} allows to make line numbers \enquote{uncopyable} but is not critical so if it is not found we provide fallbacks.}
\end{multicols}\vspace*{-.75\baselineskip}

    All of those package should be part of usual \LaTeX{} distribution. 
    
\section{References}
\subsection{Accepted Parameters}

\newenvironment{argument}[2][]{\medskip\par\noindent\phantomsection\llap{\small\texttt{#2}~}\textit{#1package option}\label{arg:#2}\\*\noindent\ignorespaces}{}


\begin{argument}{print}
    This option tries to save ink by not highlighting the background of code blocks and by making certain keywords bold or italic to ensure readability even in monochrome prints.
    See also \argref{digital}{digital} for the complementary option.
\end{argument}

\begin{argument}[default, ]{digital}
    This is the complementary option to \argref{print}{print} and is the default.
\end{argument}

\begin{argument}[default, ]{guardspace}
    This option allows you to configure a minimum number of lines that should be kept together, which you can define with the help of \cmdref{xlstguard}. If you have a border this can help avoid initial stripes. 
    
    If you explicitly want to disable this option, use \blatex{guardspace=false} when loading the package.
\end{argument}

\makeatletter
\begin{argument}{numinpar}
    This option is a shorthand for \cmdref{xlstSetLeftMargin} with the value \enquote{\@@xlst@numinpar@length}.
    It can be used to automatically indent listings and prevent numbers from protruding beyond the document margin.
\end{argument}

\begin{argument}[default, ]{hlnumbers}
    This option causes the indicative number highlighting in code blocks and is on by default, use \T{hlnumbers=false} to disable it.
\end{argument}

\begin{argument}{upshape}
    By default, our inline commands adapt to the surrounding text and may be e.g., italic. This option disables this behavior enforcing an upshape style. For a similar behavior but with font size, see \argref{inlinesize}{inlinesize}
\end{argument}

\begin{argument}{inlinesize}
    By default, our inline commands adapt to the surrounding fontsize, this option disables this behavior enforcing a fixed size. For a similar behavior but with font shape, see \argref{upshape}{upshape}. to configure the font size, see \cmdref{lstfs}.
\end{argument}

\begin{argument}{highlights}
    This option uses the \href{https://ctan.org/pkg/tikz}{tikz} package to highlight code segments. This can be used to highlight code segments in a different color or with a background. This option is currently work in progress and not easily usable.
\end{argument}

\begin{argument}[default, ]{fakeminted}
    This option causes \T{xlistings} to provide you with a replacement \blatex{minted} environment.
    To disable this behavior, use \T{fakeminted=false}.
\end{argument}

\begin{argument}{extendednums}
    This option causes numbers with underscores or the classic \T{0x}-, \T{0b}- or \T{0o}-prefixes to be highlighted as well.
\end{argument}

\begin{argument}{debug}
    This option enables debug output and causes the package to print debug information to the console.
\end{argument}

\begin{argument}[string, ]{style}
    This option selects the default style that can be set with \cmdref{xlstsetstyle}. The default is (surprise) \T{default}.
\end{argument}

\subsection{Most Important Commands}

After loading the package as described in~\cref{sec:loading-pkg},
you can load a language with \cmdref{LoadLanguages}:
\begin{minted}{latex}
\LoadLanguages{la:c:tex,ja:c:va,ba:c:sh,js:c:on}
\end{minted}
This uses the opinionated language definitions in the \T{langs/} folder and provides them to you in a set of environments and commands:
\begin{itemize}[nosep]
    \item As \blatex{\\begin\{<lang>\}}, \blatex{\\begin\{<lang>*\}}, and \blatex{\\begin\{plain<lang>\}} environments. For example, with the before loaded languages you have:
\begin{minted}{latex}
\begin{java}
...
\end{java}
\end{minted}
    \item As \blatex{\\c<lang>\{<code>\}} and \blatex{\\b<lang>\{<code>\}} commands, e.g., \clatex{\\cjava\{int i\}} and \blatex{\\bjava\{int i\}}.
    \item As \blatex{\\i<lang>\{<file>\}} command to include a file.
    \item As \blatex{\\begin\{minted\}\{<lang>\}} environment.
\end{itemize}

All inline commands (\blatex{\\c<lang>}, \blatex{\\b<lang>}, \blatex{\\i<lang>}) have an advantage in that they can be used as arguments to other macros, \textit{but} you have to escape things like backslashes or curly braces with an additional backslash.

If you want to register special keywords or any other configuration for an existing, and loaded, language, you can use \cmdref{xlstlangoverride}:

\xlstsetstyle{plain number}
\begingroup
\xlstblacklistlinenumbers{4}
\begin{minted}[autogobble]{latex}
        \xlstlangoverride{latex}{
            morekeywords=[5]{\\commandA, \\commandB},
            lineskip=42pt,
            % ...
        }
\end{minted}
\endgroup

As you can see, the sample above skips a given line number, which we achieve by using the following command: \blatex{\\xlstblacklistlinenumbers\{4\}}.
The effects of all of these commands are local, so you can use them in a group (e.g., with \blatex{\\begingroup} and \blatex{\\endgroup} or an opening and closing brace) to limit their effect.

Additionally, if you look at the source of this documentation, the source of the example above is actually indented. We remove the leading spaces with the new \T{autogobble} option (provided by the \href{https://ctan.org/pkg/lstautogobble}{lstautogobble} package).
If you want to add new literates for whatever reason, you can use the new listings key \T{add to literate}:

\begin{minted}{latex}
\lstset{
    add to literate={sample}{\(\mathcolor{red}{\pi}\)}{1}
}
\end{minted}

\begingroup
\lstset{
    add to literate={sample}{\(\mathcolor{red}{\pi}\)}{1}
}

With this active, writing \T{sample} will result in \bjava{sample} (you can combine this with \cmdref{xlstlangoverride} to add literates to a specific language).
\endgroup

If you want to change the colors of the keywords used, you can use \cmdref{lstcolorlet} with keys such as \T{keywordA}, \T{keywordB}, \T{keywordC}, \T{numbers}, \T{literals}, \T{comments}, \T{highlight}, and \T{command}:

\begin{minted}{latex}
\lstcolorlet{comments}{orange}
\end{minted}
\begingroup
\lstcolorlet{comments}{orange}

With this, the color of comments will be changed to orange:

\begin{minted}{latex}
% sample comment 
\end{minted}
\endgroup

If you want to have complete control over the styling, you may use \cmdref{xlstDefineStyles} which works like the following:

\begin{minted}{latex}
\xlstDefineStyles{%
    {comments: \color{green}\scshape},%
}
\end{minted}

\begingroup
\xlstDefineStyles{%
    {comments: \color{green}\bfseries},%
}

Now, comments will be displayed in green and bold:
\begin{minted}{latex}
% sample comment 
\end{minted}

\textit{This command is currently rather rigid and subject to change!}


If you are interested in a flexible highlighting and animation of these code snippets, have a look at the \href{https://github.com/EagleoutIce/code-animation}{code-animation} package, which we are currently working for to be compatible with \T{xlistings}.
\endgroup



\subsection{Full Reference}

\textbf{TODO}

\end{document}
\iffalse

\subsection{Farben}

    Dieses Paket definiert einige Farben mittels \cmd{colorlet}. Wird \T{sopra-base} als geladen erkannt, wird versucht passende Farben zu setzen. \imptext{Alle folgenden Farben tragen das Präfix \T{xlst@colors@}}: \T{border}, \T{background}, \T{lst@keywordA}, \T{lst@keywordB}, \T{lst@keywordC}, \T{lst@numbers}, \T{lst@literals}, \T{lst@comments}, \T{lst@highlight}, \T{lst@linenumbers} und zuletzt \T{lst@command}, wobei die Bedeutung der einzelnen Farben aufgrund ihres Namens hervorgehen sollte (jeder Sprache steht es frei die gleich zu besprechenden Stile, die
    auf diesen Farben aufbauen, (beliebig) zu verwenden). Beispielsweise kann man die Farbe
    von Kommentaren wie folgt ändern (\cmdref{lstcolorlet}):
\begin{plainlatex}[morekeywords={[2]{\\lstcolorlet}}]
\lstcolorlet{comments}{orange}
\end{plainlatex}
    Wir erhalten(exemplarisch in Java, zuerst standard, danach mit geänderter Farbe):
    \begingroup
\begin{java}
// Vor der Neuzuweisung der Farbe
public static void main(...){ /* ... */ }
\end{java}
\lstcolorlet{comments}{orange}
\begin{java}
// Nach Neuzuweisung der Farbe
public static void main(...){ /* ... */ }
    \end{java}
    \endgroup
 Es ist möglich jede Farbe (nach \T{xcolor}) zu verwenden (siehe auch \cmdref{lstcolordef}).

\subsection{Stile}
    Zur Hervorhebung der einzelnen Segmente im Listing werden den Farben entlehnte Stile definiert, die ebenfalls frei verändert werden können. \notetext{Entwicklernotiz, die
    Stile sollten bei einer kompletten Neudefintion mithilfe des Befehls \cmd{xlst@list@define@styles}
    definiert werden. Sonst ist das Präfix, das im Folgenden verwendet werden soll in
    \cmd{xlst@lst@style@prefix} abgespeichert.}\par{}
    Alle im folgenden aufgeführten Befehle tragen (sofern nicht modifiziert) das Präfix
    \T{xlst@styles@lst@} und können (ohne dieses Präfix) mittels \cmdref{solGet} beziehungsweise \cmdref{solGetStyle} abgefragt werden. Einzelne Sprachen können
    weitere Befehle und Stile fordern und hinzufügen, davon sollte aber nur im Notfall gebraucht gemacht werden (so fügt die Sprache \T{Latex} den Stil \T{command} hinzu. Für mehr Informationen, siehe: \jmark[Abschnitt \ref{sec:vordefinierteSprachen} (Vordefinierte Sprachen)]{sec:vordefinierteSprachen})\par{}
    Es gibt die folgenden Stile: \T{keywordA}, \T{keywordB}, \T{keywordC}, \T{keywordD}, \T{numbers}, \T{linenumbers}, \T{literals}, \T{comments}, \T{highlight}, \T{command} und \T{basic}, wobei letzterer jedem Stil vorangestellt wird. Die Neudefinition eines Stils sollte
    nicht leichtfertig getätigt werden und benötigt deswegen etwas mehr Aufwand. Hier ein Beispiel:
\begin{plainlatex}[morekeywords={[5]{\\xlst@list@define@styles,\\scshape}}]
{\makeatletter
    \xlst@list@define@styles{%
        {comments: \color{xlst@colors@lst@comments}\scshape}%
    }
}
\end{plainlatex}
Wir erhalten(exemplarisch in Java):
    \begingroup
\begin{java}
// Vor der Neuzuweisung des Stils
public static void main(...){ /* ... */ }
\end{java}
{\makeatletter
    \xlst@list@define@styles{%
        {comments: \color{xlst@colors@lst@comments}\scshape}%
    }
}
\begin{java}
// Nach Neuzuweisung des Stils
public static void main(...){ /* ... */ }
    \end{java}
    \endgroup
    \notetext{Mehrere Stile können durch Kommas voneinander abgetrennt in einem Zug definiert werden und überleben Gruppen.}
{\makeatletter
    \xlst@list@define@styles{%
        {comments: \color{xlst@colors@lst@comments}}%
    }
}

\section{Befehle- und Umgebungen}

Es gilt zu beachten, dass das Präfix \T{env@} nur auf die Natur einer Umgebung hinweist und nicht zum eigentlichen Bezeichner zuzuordnen ist!

% TODO: mke command for lengths

\subsection{Die Umgebungen zum Setzen}

\notetext{Diese Umgebungen werden für jede geladene Sprache erstellt. Jede in diesem Paket enthaltene Sprache folgt der Regel, dass sie mit einem \say{l} (für \textbf{l}anguage) beginnt und dann ein Großbuchstabe folgt. Möchte man also \T{java} aus irgendeinem Grunde
manuel in \env{lstlisting} oder so verwenden, so heißt die Sprache hier \T{lJava}.}

\begin{environment}{<Sprache>}{\optArg{Stildefinitionen}}
    Setzt den eingeschlossenen Code in einer durch \cmdref{solLoadLanguage} \T{<Sprache>}. \notetext{Also, wichtig: \env{<Sprache>} selbst, gibt es nicht.}.
    Beispiel:\smallskip\\ % gosh i should have implemented the ltx preview
    \begin{latex}
\begin{java}
// Ein Kommentar!!
public static void main(...){ /* ... */ }
\end{java}
    \end{latex}
    Ergibt (\notetext{Meta-Kommentar: Die Anzeige des \LaTeXe-Codes wurde mit \env{latex} vollzogen}):
\begin{java}
// Ein Kommentar!!
public static void main(...){ /* ... */ }
\end{java}
    Bei \T{Stildefinitionen} handelt es sich übrigens um Definitionen für \T{listings}.
\end{environment}

\begin{environment}{<Sprache>*}{\optArg{Stildefinitionen}}
    Analog zu \envref{<Sprache>}, allerdings ohne Zeilennummern.
\end{environment}

\begin{environment}{plain<Sprache>}{\optArg{Stildefinitionen}}
    Analog zu \envref{<Sprache>}, allerdings ohne Zeilennummern und Hintegrund.
\end{environment}

\begin{command}{c<Sprache>}{\optArg{Stildefinitionen}\manArg{Code}}
    Setzt den \T{Code} in einer Zeile in der jeweiligen Sprache. \notetext{Wichtige Bemerkung: Da \LaTeXe{} nicht davon abgehalten werden kann die Zeichen zumindest einfach zu expandieren, müssen Zeichen wie ein Anführungszeichen oder geschwungene Klammern mittels eines Backslash escaped werden. Dies kann weiter dafür sorgen, dass in diesem
    Fall Leerfelder verloren gehen. Ein solches Leerfeld kann durch \say{\T{:ws:}}, ein \jmark[Literate]{mrk:Literates}, forciert werden.}\\
    Beispiel:
    \begin{plainlatex}
Dies ist ein Test für \cjava{public static void main} - hat er geklappt?
    \end{plainlatex}
    Ergibt: \\
Dies ist ein Test für \cjava{public static void main} - hat er geklappt?
\end{command}

\begin{command}{b<Sprache>}{\optArg{Stildefinitionen}\manArg{Code}}
   Analog zu \cmdref{c<Sprache>}, allerdings wird kein Hintergrund und kein Rahmen gesetzt, was dafür sorgt, dass auch Zeilenumbrüche kein Problem sind.
\end{command}

\begin{command}{i<Sprache>}{\optArg{Stildefinitionen}\manArg{Dateipfad}}
    Bindet die Datei in \T{Dateipfad} in das Dokument ein, wobei \envref{<Sprache>} zum Highlighting verwendet wird.
 \end{command}

\subsection{Sprachverwaltung}

\begin{command}{solStyles}{}
    Zeigt die Formatierung und Farbe der verschiedenen Stile an und ist wohl nur zu Testzwecken von nutzen. So ergibt \cmd{solStyles}: \xlstStyles
\end{command}

\begin{command}{solLanguageSearchPath}{}
    Dieser Befehl enthält alle Pfade an denen \cmdref{solLoadLanguage} nach sprachen suchen soll. Dies ist überflüssig, wenn sich die Sprache im \T{texmf}-Baum des Systems befindet. Die Pfade können durch \cmd{renewcommand} aktualisiert werden. Beispiel:
\begin{plainlatex}
\renewcommand{\xlstLanguageSearchPath}{{OrdnerA/}{OrdnerB/UnterordnerA/}{/home/}}
\end{plainlatex}
    \notetext{Im lokalen Verzeichnis in dem \bbash{pdflatex} aufgerufen wird, wird immer gesucht! Sollte es mehrere Dateien mit gleichem Namen geben, so hat die Datei im lokalen Verzeichnis die höchste Priorität und anschließend werden die Ordner in der Reihenfolge durchsucht, bis die Datei das erste mal gefunden wurde.}
\end{command}

\begin{command}{solLoadLanguage}{\manArg{LanguageName}}
    Sucht und Lädt Sprachen in \T{solLanguageSearchPath}. Es können mehrere Sprachen mittels Kommaseparierung angegeben werden. Der Befehl kann mehrfach aufgerufen werden, erlaubt aber nicht, die selbe Sprache mehrfach zu laden. Die Sprache muss in einer Datei mit dem Namensschema \T{language\_<LanguageName>.cfg} definiert sein. Also zum Beispiel \T{language\_java.cfg}. Wie die Definition einer Sprache auszusehen hat, erklärt \cmdref{RegisterLanguage}. \notetext{Die Sprache \T{lVoid} mit \T{void} wird immer geladen und kann als Verbatim-Umgebung verwendet werden.}
\end{command}

\begin{command}{RegisterLanguage}{\optArg{Aliasse}\manArg{Sprache}\manArg{LanguageName}}
    Registriert eine Sprache mit dem Namen \T{LanguageName} als \T{Sprache}. Dieser Aufruf muss von jeder Konfiguration selbst durchgeführt werden. \par{}
    Im folgenden Beispiel gilt es die Sprache \T{Rubberduck} zu definieren und zu laden. In die Konfigurationsdatei schreiben wir:
\begin{latex}
\lstdefinelanguage{rubberduck}{
    comment=[l]{\#},
    morekeywords = {Quack, new},
    morekeywords = [2]{Duck}
}
\RegisterLanguage{rubberduck}{rubberduck}
\end{latex}
    Also eine Sprache, die Raute für Kommentare verwendet, die beiden Schlüsselbegriffe \T{Quack} und \T{new} besitzt sowie noch ein \say{keywordB} (\notetext{so Formal ist die Sprache jetzt ja auch nicht :)}): \T{Duck}
    Im Dokument laden wir die Sprache mit:
\begin{latex}
\xlstLoadLanguage{rubberduck}
\end{latex}
    Nun können wir die Umgebungen wie \envref{<Sprache>} auch mit \T{rubberduck} verwenden:
\begin{latex}[morekeywords={[3]{rubberduck}}]
\begin{rubberduck}
Duck jens = new Duck(); # Eine neue Ente
Quack jens ::{
    Quack Quack, Quack Quack
    Quack Quack. # Entisch, es ist so simpel
}
\end{rubberduck}
\end{latex}
Ergibt:
\begin{rubberduck}
Duck jens = new Duck(); # Eine neue Ente
Quack jens ::{
    Quack Quack, Quack Quack
    Quack Quack. # Entisch, es ist so simpel
}
\end{rubberduck}
    \notetext{Die Befehle wie \cmdref{c<Sprache>}, also \cmd{crubberduck} funktionieren natürlich auch: \crubberduck{Duck jens}.}

    Ist \secondargref{nofakeminted}{fakeminted} nicht gesetzt, so haben die Aliasse aktuell keine Bedeutung.
    Werden sonst keine Aliasse übergeben, so wird automatisch ein mapping (\cmdref{solmintedmap}) von \T{Sprache} zu \T{LanguageName} erstellt. Andernfalls, wird für jedes (Komma-separierte) Alias, ein Mapping zu \T{LanguageName} kreiert, sodass es mit \envref{minted} verwendbar ist.
\end{command}

\subsection{Allgemeine Befehle}

\begin{command}{thesolversion}{}
    Liefert die aktuelle Version des Pakets. So ergibt: \cmd{thesolversion}: \thesolversion\\
    \notetext{Hinweis: über \blatex{\\value\{solversion\}} lässt sich
    die Version als $4$-stellige Nummer erhalten: \arabic{solversion}.}
\end{command}

\begin{command}{solSetLeftMargin}{\manArg{Länge}}
    Setzt den Abstand der linken Seite des Listings und kann so dafür sorgen, dass zum
    Beispiel auch die Zeilennummern nicht über den Dokumentrand hinausragen. Beispiel:
\begin{latex}[morekeywords={[5]{\\xlstSetLeftMargin}}]
\xlstSetLeftMargin{15pt}
\begin{java}
// Ein Kommentar!!
public static void main(...){ /* ... */ }
\end{java}
\end{latex}
Ergibt: {
\xlstSetLeftMargin{15pt}
\begin{java}
// Ein Kommentar!!
public static void main(...){ /* ... */ }
\end{java}
}
\end{command}

\begin{command}{solSetRightMargin}{\manArg{Länge}}
    Setzt den Abstand der rechten Seite des Listings, sonst analog zu \cmdref{solSetLeftMargin}.
\end{command}

\begin{command}{solSetNumSep}{\manArg{Länge}}
    Setzt den Abstand der Zeilennummern zum Code.
\end{command}

\begin{command}{solSetFrameRule}{\manArg{Länge}}
    Setzt die Dicke der Umrandung (Standard ist hier \bvoid{0.75pt}, also nicht übertreiben :P).
\end{command}

\begin{command}{T}{\manArg{Text}}
    Setzt den \T{Text} in \T{Schreibmaschinenschrift}.
\end{command}

\begin{command}{solGetStyle}{\manArg{StilName}}
    Expandiert zum Stil von \T{StilName} wie ihn das Paket auch selbst verwendet.
\end{command}

\begin{command}{solGet}{\manArg{StilName}\manArg{Text}}
    Setzt \T{Text} im Stil von \T{StilName} wie ihn das Paket auch selbst verwendet. So zum Beispiel:\\*\bjava{\\xlstGet\{comments\}\{Hallo Welt\}} ergibt: \xlstGet{comments}{Hallo Welt}.
\end{command}

\begin{command}{solNumFs}{\manArg{size}}
    Ändert die Schriftgröße der Zeilennummern separat.
\end{command}

\begin{command}{lstfs}{\optStar\optArg{linespread}\manArg{size}}
    Passt die Schriftgröße des Codes wie auch die der Zeilennummern an. Bei letzteren wird versucht eine geeignete Größe zu finden.
    Die Variante mit Stern hat \textit{keine} Angabe der \T{size}, in diesem Fall wird automatisch die aktuelle Schriftgröße gesetzt.
\begin{latex}[morekeywords={[5]{\\lstfs}}]
\begin{java}
public static int add(int a, int b){ return a + b; }
\end{java}
{\lstfs{5}
\begin{java}
public static int add(int a, int b){ return a + b; }
\end{java}
}
\begin{java}
public static int add(int a, int b){ return a + b; }
\end{java}
{\tiny
\begin{java}
public static int add(int a, int b){ return a + b; }
\end{java}
}
\end{latex}
Ergibt:
\begin{java}
public static int add(int a, int b){ return a + b; }
\end{java}
{\lstfs{5}
\begin{java}
public static int add(int a, int b){ return a + b; }
\end{java}
}
\begin{java}
public static int add(int a, int b){ return a + b; }
\end{java}
{\tiny
\begin{java}
public static int add(int a, int b){ return a + b; }
\end{java}
}
\end{command}

\begin{command}{lstcolorlet}{\manArg{lstcolor}\manArg{normalcolor}}
    Weißt eine Farbe mittels \cmd{colorlet} zu. Hierbei darf \emph{lstcolor} nur einen der folgenden Werte annehmen: \emph{keywordA}, \emph{keywordB}, \emph{keywordC}, \emph{numbers}, \emph{literals}, \emph{comments}, \emph{highlight} oder \emph{command}.
\end{command}

\begin{command}{lstcolordef}{\optArg{mode=RGB}\manArg{lstcolor}\manArg{colorcode}}
    Weißt eine Farbe mittels \cmd{definecolor} zu. Hierbei darf \emph{lstcolor} nur einen der folgenden Werte annehmen: \emph{keywordA}, \emph{keywordB}, \emph{keywordC}, \emph{numbers}, \emph{literals}, \emph{comments}, \emph{highlight} oder \emph{command}. Das erste Argument gibt den Modus nach \emph{xcolor} an.
\end{command}

\begin{command}{solguard}{\manArg{Zeilen}}
    Mit der Paketoption \argref{guardspace}{noguardspace} kann so angegeben werden, wie viele Zeilen geschützt werden müssen.
    Der Standard ist: \glqq{}{\makeatletter\@xlst@guard@default}\grqq. Hinweis: Dies betrifft nur die Zeilen für den Beginn. Die letzte Zeile wird aktuell immer mit \cmd{nobreak} gesichert.
\end{command}

\begin{command}{solblacklistlinenumbers}{\optStar\manArg{Zeilennummern}}
    Eine kommaseparierte Liste an Zeilennummern, welche nicht angezeigt werden sollen. Die Sperre erfolgt lokal, mittels \TeX-Gruppen kann das blacklisting demnach kontrolliert werden.
    Mit \cmdref{solblacklistisghost} \cmdref{solblacklistispresent} kann kontrolliert werden, ob die ausgeblendeten Zeilennummern aus der Zählung entfernt werden sollen.
    Dies kann auch über den optionalen Stern geregelt werden (\cmdref{solblacklistisghost} ist standard, mit Stern wird \cmdref{solblacklistispresent} ausgeführt). Wiederholte Aufrufe fügen die zu versteckenden Zeilen den Bestehenden hinzu.
\begin{latex}[morekeywords={[5]{\\xlstblacklistlinenumbers}}]
\xlstblacklistlinenumbers{3}
\begin{java}
public static void main(String[] args){
    System.out.println("Hello World");
    System.out.println("Ghooost!");
}
\end{java}
\end{latex}
Ergibt: {
\xlstblacklistlinenumbers{3}
\begin{java}
public static void main(String[] args){
    System.out.println("Hello World");
    System.out.println("Ghooost!");
}
\end{java}
}
Im Gegensatz dazu, mit Stern:
\begin{latex}[morekeywords={[5]{\\xlstblacklistlinenumbers}}]
\xlstblacklistlinenumbers!**!*{3}
\begin{java}
public static void main(String[] args){
    System.out.println("Hello World");
    System.out.println("Ghooost!");
}
\end{java}
\end{latex}
Ergibt: {
\xlstblacklistlinenumbers*{2,3}
\begin{java}
public static void main(String[] args){
    System.out.println("Hello World");
    System.out.println("Ghooost!");
}
\end{java}
}
\end{command}

\begin{command}{solblacklistisghost}{}
    Mit \cmdref{solblacklistlinenumbers} ausgeblendete Zeilen werden auch nicht gezählt.
\end{command}

\begin{command}{solblacklistispresent}{}
    Mit \cmdref{solblacklistlinenumbers} ausgeblendete Zeilen werden dennoch gezählt.
\end{command}

\subsection{Minted spezifisches}

Ist die Paketoption \secondargref{nofakeminted}{fakeminted} gesetzt, so existieren weiter die folgenden Befehle und Umgebungen.

\begin{environment}{minted}{\optArg{lising-Args}\manArg{language}}
    Setzt den eingeschlossenen Text im Stil von \cmdref{solsetmintedstyle}, wobei die Sprache wie in \cmdref{solmintedmap} konfiguriert, abgebildet wird.
\end{environment}

\begin{command}{solmintedmap}{\manArg{from}\manArg{to}}
    Erstellt oder überschreibt eine Sprachabbildung für \envref{minted}.
    Durch \cmdref{RegisterLanguage} wird dies automatisch vollzogen (allerdings direkt durch interne Mechanismen).
\end{command}

\begin{command}{solsetmintedstyle}{\manArg{default/nonumber/plain/plain number}}
    Aktualisiert den von \envref{minted} zu verwendenden Stil.
\end{command}

\clearpage
\appendix
\section{Sprachdefinition}

\subsection{Vordefinierte Sprachen}
\label{sec:vordefinierteSprachen}

Die hier definierten Sprachdateien befinden sich im Unterordner \T{Languages} und können auch nur als Basis für eine eigene Sprachdefinition genutzt werden. Sie enthalten schon einige Beispielwerte, so wie ich sie verwende, können aber in der Hinsicht auch komplett
auf \say{Blank} zurückgesetzt werden. \notetext{Hinweis: Die Definition der Sprachen verläuft, Überraschung, analog zu Lilly (\url{https://github.com/EagleoutIce/LILLY}, nur die geforderte Dateinennung und die weiteren Möglichkeiten unterscheiden sich.), also können auch
diese Sprachdefinitionen als Beispiel herangezogen werden. Weitere Informationen bietet die Dokumentation zu \T{listings}.}

\paragraph{Java (\T{lJava}, \T{java})}

Definiert in diesem Kontext nichts besonderes.

\paragraph{Bash (\T{lBash}, \T{bash})}

Definiert in diesem Kontext nichts besonderes.

\paragraph{Json (\T{lJson}, \T{json})}

Definiert in diesem Kontext nichts besonderes.

\paragraph{latex (\T{lLatex}, \T{latex})}

Definiert \T{command} als zusätzlichen Stil für Befehle.

\paragraph{XML (\T{lXml}, \T{xml})}

Definiert in diesem Kontext nichts besonderes.

\paragraph{Rubberduck (\T{lRubberduck}, \T{rubberduck})}

Definiert in diesem Kontext nichts besonderes. Auch keine echte Sprache :P.

\subsection{Literates}
\label{mrk:Literates}
Im Kontext verschiedener Programmiersprachen kam bald der Wunsch auf verschiedene Symbole entsprechend einfach Setzen zu können. Die Ersetzungsregeln werden grundlegend geladen, wenn das Paket geladen wird, allerdings können einzelne Programmiersprachen (so wie Java mit seinen Escape-Sequenzen) noch eigene definieren. Die Ersetzungsregeln ermöglichen es, neben Umlauten auch Symbole einzubinden. Die Ersetzungsregeln werden nicht über eine Liste gehandhabt und sind ebenso vielfältig wie es die Bedürfnisse erfordern. Im Folgenden eine Auflistung aller wichtigen Ersetzungsregeln:
\begin{multicols}{4}
    \begin{description}
        \newcommand{\lstshowcmd}[2][]{\edef\tmpdocsolb{\noexpand\lstinline[#1]!#2!}\tmpdocsolb}
        \foreach \x in {:bs:,:bmath:,:emath:,:dollar:,:space:,:ws:,:cdots:,:cdot:,:ldots:,:c:,:float:,:exp:,:yields:,:lan:,:ran:,:bcmd:,:star:,:percent:} {
            \item[{\T{\x}}] \say{\lstshowcmd{\x}}
        }
    \end{description}
\end{multicols}
\notetext{Hinweis: Ja, \T{:c:} expandiert zu nichts. Es kann als Trenner verwendet werden. Allerdings sollte dies nicht notwendig sein,
da das Paket von Version \T{1.0.14} an automatische Maßnahmen ergreift: \bjava{14 + 3 = 17 * (4 - 3) :yields: :bmath: 17:emath::ldots: !}}

\clearpage

\section{Beispiele und Tests}

Die Codes hierzu finden sich am Ende des Quellcodes der Dokumentation:

So gilt: \cjava{public static void main(:ldots:)} definiert die \bjava{main}-Methode\ldots\blindtext[1]
\begin{plainjava}
// Dies ist sehr guter Code
public static void main(String[] args){
    System.out.println("Hallo Welt");
    System.out.print(2+3*4-5)
}
\end{plainjava}

\begin{java}
import java.util.Scanner;

/**
 *  @file ean.java
 *  @author Florian Sihler, Raphael Straub
 *  @brief Das Programm berechnet anhand den ersten 12 Stellen einer EAN die zugehörige Prüfsumme
 *  @version 1.0
 *  @date 2.11.2018
 */

/**
 * @class ean
 * @brief Die Hauptklasse des Programms, in ihr 'lebt' der gesamte Code
 */
public class ean {
    ///@brief der Einstiegspunkt des Programms
    public static void main(String[] args){
        Scanner scanner = new Scanner(System.in); //Initialisierung
        int sum = 0; boolean toggle = false;
        for(int i = 0; i < 12; i++){ //Einlesen der 12 Zahlen
                                         /* abwechselnd *1 und *3 */
            sum += (scanner.nextInt() * ((toggle)?3:1)); //Summieren der Zahl
            toggle = !toggle; //Invertieren des Multiplikators
        }
        System.out.println( (10 - (sum % 10)) % 10 );
        // Gibt die Prüfsumme als den Abstand zum nächsten Vielfachen von 10
        scanner.close();
    }
}
\end{java}

\begin{xml*}[morekeywords={[3]{collection,book,title,authorlist,description,date,author}}]
<?xml version="1.0" encoding="UTF-8"?>
<collection>
    <book category="Roman">
        <!-- Ich bin ein Kommentar -->
        <title>Tolles Buch</title>
        <authorlist>
            <author>Netter Mensch</author>
            <author>Noch netterer Mensch</author>
        </authorlist>
        <description />
        <date> heute, wann sonst? </date>
    </book>
    <book category="Gute Bücher">
        ...
    </book>
</collection>
\end{xml*}

\begin{json}
{"menu": {
  "id": "file",
  "value": "File",
  "popup": {
    "menuitem": [
      {"value": "New", "onclick": "CreateNewDoc()"},
      {"value": "Open", "onclick": "OpenDoc()"},
      {"value": "Close", "onclick": "CloseDoc()"}
    ]
  }
}}
\end{json}

\begin{void}
Einfach nur eine Verbatim-Ausgabe. 1234 Hallo42, Hallo 42, Hallo 42
\end{void}

\end{document}
\fi 